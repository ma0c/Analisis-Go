\documentclass{article}
\usepackage[utf8]{inputenc}
\usepackage[spanish]{babel}
\usepackage{amssymb}
\usepackage{hyperref}
\usepackage{listings}
\lstset{}
\author{Mauricio Collazos \\
\href{mailto:ma0@contraslash.com}{ma0@contraslash.com}}
\title{Análisis de un lenguaje de programación: Go}
\begin{document}
\maketitle
\section{Introducción}
Go es un lenguaje de programación de código abierto liberado al público el 10 de Noviembre de 2009 y con primera versión estable liberada el 28 de marzo de 2012 . Las principales motivaciones al diseñar el lenguaje de programación se basaron en crear un lenguaje de compilación rápida, tipado estático, con un recolector de basura eficiente y diseñado para las la construcción de sistemas en máquinas multinúcleo. \cite{FrequentlyLanguage}.

En este documento se explorarán los modelos de programación soportados por el lenguaje de programación Go, incluyendo el Modelo de Programación Declarativa, el Modelo de Programación Concurrente Declarativa, el Modelo de Programación con Estado, el Modelo de Programación Concurrente por paso de mensajes, el Modelo de Programación Orientado a objetos y por último el Modelo de Programación Relacional.

\section{Modelo de Programación Declarativa}
https://medium.com/@geisonfgfg/functional-go-bc116f4c96a4

\bibliographystyle{plain} 
\bibliography{mendeley_v2.bib}
\end{document}
